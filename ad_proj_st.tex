\documentclass[]{article}
\usepackage{lmodern}
\usepackage{amssymb,amsmath}
\usepackage{ifxetex,ifluatex}
\usepackage{fixltx2e} % provides \textsubscript
\ifnum 0\ifxetex 1\fi\ifluatex 1\fi=0 % if pdftex
  \usepackage[T1]{fontenc}
  \usepackage[utf8]{inputenc}
\else % if luatex or xelatex
  \ifxetex
    \usepackage{mathspec}
  \else
    \usepackage{fontspec}
  \fi
  \defaultfontfeatures{Ligatures=TeX,Scale=MatchLowercase}
\fi
% use upquote if available, for straight quotes in verbatim environments
\IfFileExists{upquote.sty}{\usepackage{upquote}}{}
% use microtype if available
\IfFileExists{microtype.sty}{%
\usepackage{microtype}
\UseMicrotypeSet[protrusion]{basicmath} % disable protrusion for tt fonts
}{}
\usepackage[margin=1in]{geometry}
\usepackage{hyperref}
\hypersetup{unicode=true,
            pdftitle={PROJECTE ANÀLISI DE DADES: Entrada Turística Espanya},
            pdfauthor={David Anglada Rotger i Andreu Huguet Segarra},
            pdfborder={0 0 0},
            breaklinks=true}
\urlstyle{same}  % don't use monospace font for urls
\usepackage{color}
\usepackage{fancyvrb}
\newcommand{\VerbBar}{|}
\newcommand{\VERB}{\Verb[commandchars=\\\{\}]}
\DefineVerbatimEnvironment{Highlighting}{Verbatim}{commandchars=\\\{\}}
% Add ',fontsize=\small' for more characters per line
\usepackage{framed}
\definecolor{shadecolor}{RGB}{248,248,248}
\newenvironment{Shaded}{\begin{snugshade}}{\end{snugshade}}
\newcommand{\KeywordTok}[1]{\textcolor[rgb]{0.13,0.29,0.53}{\textbf{#1}}}
\newcommand{\DataTypeTok}[1]{\textcolor[rgb]{0.13,0.29,0.53}{#1}}
\newcommand{\DecValTok}[1]{\textcolor[rgb]{0.00,0.00,0.81}{#1}}
\newcommand{\BaseNTok}[1]{\textcolor[rgb]{0.00,0.00,0.81}{#1}}
\newcommand{\FloatTok}[1]{\textcolor[rgb]{0.00,0.00,0.81}{#1}}
\newcommand{\ConstantTok}[1]{\textcolor[rgb]{0.00,0.00,0.00}{#1}}
\newcommand{\CharTok}[1]{\textcolor[rgb]{0.31,0.60,0.02}{#1}}
\newcommand{\SpecialCharTok}[1]{\textcolor[rgb]{0.00,0.00,0.00}{#1}}
\newcommand{\StringTok}[1]{\textcolor[rgb]{0.31,0.60,0.02}{#1}}
\newcommand{\VerbatimStringTok}[1]{\textcolor[rgb]{0.31,0.60,0.02}{#1}}
\newcommand{\SpecialStringTok}[1]{\textcolor[rgb]{0.31,0.60,0.02}{#1}}
\newcommand{\ImportTok}[1]{#1}
\newcommand{\CommentTok}[1]{\textcolor[rgb]{0.56,0.35,0.01}{\textit{#1}}}
\newcommand{\DocumentationTok}[1]{\textcolor[rgb]{0.56,0.35,0.01}{\textbf{\textit{#1}}}}
\newcommand{\AnnotationTok}[1]{\textcolor[rgb]{0.56,0.35,0.01}{\textbf{\textit{#1}}}}
\newcommand{\CommentVarTok}[1]{\textcolor[rgb]{0.56,0.35,0.01}{\textbf{\textit{#1}}}}
\newcommand{\OtherTok}[1]{\textcolor[rgb]{0.56,0.35,0.01}{#1}}
\newcommand{\FunctionTok}[1]{\textcolor[rgb]{0.00,0.00,0.00}{#1}}
\newcommand{\VariableTok}[1]{\textcolor[rgb]{0.00,0.00,0.00}{#1}}
\newcommand{\ControlFlowTok}[1]{\textcolor[rgb]{0.13,0.29,0.53}{\textbf{#1}}}
\newcommand{\OperatorTok}[1]{\textcolor[rgb]{0.81,0.36,0.00}{\textbf{#1}}}
\newcommand{\BuiltInTok}[1]{#1}
\newcommand{\ExtensionTok}[1]{#1}
\newcommand{\PreprocessorTok}[1]{\textcolor[rgb]{0.56,0.35,0.01}{\textit{#1}}}
\newcommand{\AttributeTok}[1]{\textcolor[rgb]{0.77,0.63,0.00}{#1}}
\newcommand{\RegionMarkerTok}[1]{#1}
\newcommand{\InformationTok}[1]{\textcolor[rgb]{0.56,0.35,0.01}{\textbf{\textit{#1}}}}
\newcommand{\WarningTok}[1]{\textcolor[rgb]{0.56,0.35,0.01}{\textbf{\textit{#1}}}}
\newcommand{\AlertTok}[1]{\textcolor[rgb]{0.94,0.16,0.16}{#1}}
\newcommand{\ErrorTok}[1]{\textcolor[rgb]{0.64,0.00,0.00}{\textbf{#1}}}
\newcommand{\NormalTok}[1]{#1}
\usepackage{graphicx,grffile}
\makeatletter
\def\maxwidth{\ifdim\Gin@nat@width>\linewidth\linewidth\else\Gin@nat@width\fi}
\def\maxheight{\ifdim\Gin@nat@height>\textheight\textheight\else\Gin@nat@height\fi}
\makeatother
% Scale images if necessary, so that they will not overflow the page
% margins by default, and it is still possible to overwrite the defaults
% using explicit options in \includegraphics[width, height, ...]{}
\setkeys{Gin}{width=\maxwidth,height=\maxheight,keepaspectratio}
\IfFileExists{parskip.sty}{%
\usepackage{parskip}
}{% else
\setlength{\parindent}{0pt}
\setlength{\parskip}{6pt plus 2pt minus 1pt}
}
\setlength{\emergencystretch}{3em}  % prevent overfull lines
\providecommand{\tightlist}{%
  \setlength{\itemsep}{0pt}\setlength{\parskip}{0pt}}
\setcounter{secnumdepth}{5}
% Redefines (sub)paragraphs to behave more like sections
\ifx\paragraph\undefined\else
\let\oldparagraph\paragraph
\renewcommand{\paragraph}[1]{\oldparagraph{#1}\mbox{}}
\fi
\ifx\subparagraph\undefined\else
\let\oldsubparagraph\subparagraph
\renewcommand{\subparagraph}[1]{\oldsubparagraph{#1}\mbox{}}
\fi

%%% Use protect on footnotes to avoid problems with footnotes in titles
\let\rmarkdownfootnote\footnote%
\def\footnote{\protect\rmarkdownfootnote}

%%% Change title format to be more compact
\usepackage{titling}

% Create subtitle command for use in maketitle
\newcommand{\subtitle}[1]{
  \posttitle{
    \begin{center}\large#1\end{center}
    }
}

\setlength{\droptitle}{-2em}

  \title{PROJECTE ANÀLISI DE DADES: Entrada Turística Espanya}
    \pretitle{\vspace{\droptitle}\centering\huge}
  \posttitle{\par}
    \author{David Anglada Rotger i Andreu Huguet Segarra}
    \preauthor{\centering\large\emph}
  \postauthor{\par}
      \predate{\centering\large\emph}
  \postdate{\par}
    \date{17/5/2019}


\begin{document}
\maketitle

{
\setcounter{tocdepth}{2}
\tableofcontents
}
\section{Introducció}\label{introduccio}

\section{Identificació}\label{identificacio}

\subsection{Representació gràfica de les
dades}\label{representacio-grafica-de-les-dades}

Un cop feta la representació de les dades, s'observa una clara tendència
creixent. Tot i així, aquesta tendència no és constant, ja que és menys
pronunciada entre els anys 2000 i 2010, fins i tot amb una petita
baixada entre els anys 2007i 2010 i sembla que es pronuncia a partir de
l'any 2011.

Pel que fa a la variància, s'observa que va augmentant a mesura que
augmenta la mitjana dels valors de les dades, és a dir, a mesura que es
pronuncia la tendència creixent. És a dir, en els anys 2000-2010, la
variància és menor que en els anys 2011-2019, on el creixement augmenta.

\includegraphics{ad_proj_st_files/figure-latex/unnamed-chunk-1-1.pdf}

\subsubsection{Descomposició en components
bàsiques}\label{descomposicio-en-components-basiques}

Per poder analitzar millor les dades, es realitza la seva descomposició
en les seves components bàsiques, és a dir, el model aditiu de la serie:

\[ X_t = T_t + S_t + C_t + \omega_t \] on: - \(T_t\) és la
\textbf{tendència} de la sèrie a llarg termini. - \(S_t\) és el
\textbf{\emph{seasonal}} de la sèrie (patró repetit periòdicament amb
període constant). - \(C_t\) és el \textbf{cicle} de la sèrie (patró
repetit periòdicament amb període no constant). Aquesta part no surt
representada en la descomposició. - \(\omega_t\) és el soroll aleatòri.

\includegraphics{ad_proj_st_files/figure-latex/unnamed-chunk-2-1.pdf}

S'observa, tal i com s'havia comentat anteriorment, la clara tendència
creixent de la sèrie, amb un creixement menys pronunciat a l'inici, una
petita baixada entre els anys 20017 i 2010 i una pujada més pronunciada
més cap a l'actualitat. Pel que fa al patró estacional, observem que
durant els mesos d'estiu, el número de turistes a Espanya augmenta molt
considerablement. Aquest fet que no crida l'atenció, ja que és durant
els mesos d'estiu quan més vacanses s'agafa la gent i més aprofiten per
venir a les costes espanyoles. Durant els mesos de tardor-hivern,
observem que el número de turistes cau en picat.

\subsection{Transformació de les
dades}\label{transformacio-de-les-dades}

A continuació s'analitzarà la necessitat de realitzar una sèrie de
transformacions amb l'objectiu d'aconseguir estacionaritat en la nostra
sèrie temporal.

\subsubsection{Variància constant}\label{variancia-constant}

En primer lloc, s'estudiarà si es pot considerar que la variància de les
dades sigui constant en el temps. Ja s'ha comentat que a simple vista
semblava que no. Tot i així es comprova amb un plot de la variància
front la mitjana i un \emph{boxplot} de les dades cada 12 mesos (que és
la freqüència de les nostres dades).

\begin{verbatim}
## Warning in matrix(serie, nr = 12): data length [227] is not a sub-multiple
## or multiple of the number of rows [12]

## Warning in matrix(serie, nr = 12): data length [227] is not a sub-multiple
## or multiple of the number of rows [12]
\end{verbatim}

\includegraphics{ad_proj_st_files/figure-latex/unnamed-chunk-3-1.pdf}

Tal i com s'havia observat a simple vista, la variància augmenta a
mesura que agumenta la mitja. Per tant, no podem assumir variància
constant. Amb el \emph{boxplot} es confirma aquesta hipòtesis. Així
doncs, es procedeix a realitzar una transformació logarítmica de la
sèrie per homogeneïtzar la variància. Els resultats obtinguts són els
següents:

\begin{verbatim}
## Warning in matrix(logserie, nr = 12): data length [227] is not a sub-
## multiple or multiple of the number of rows [12]

## Warning in matrix(logserie, nr = 12): data length [227] is not a sub-
## multiple or multiple of the number of rows [12]
\end{verbatim}

\includegraphics{ad_proj_st_files/figure-latex/unnamed-chunk-4-1.pdf}

S'observa que la variància s'ha homogeneïtzat, és a dir, ja es pot
considerar constant.

\subsubsection{Patró estacional}\label{patro-estacional}

En segon lloc, s'estudiarà l'existència d'un patró estacional en les
nostres dades. En cas que hi sigui present, es realitzarà una
diferenciació d'ordre 12, és a dir,

\[ W_t = X_t - X_{t-12} = (1 - B^{12})X_t \] on \(B\) és el
\emph{backshift operator}, per eliminar aquest patró. Es realitza un
\emph{monthplot} per comprovar-ne l'existència.

\begin{Shaded}
\begin{Highlighting}[]
\KeywordTok{monthplot}\NormalTok{(logserie)}
\end{Highlighting}
\end{Shaded}

\includegraphics{ad_proj_st_files/figure-latex/unnamed-chunk-5-1.pdf}

Tal i com s'havia comenta, s'observa una clara pujada de la presència de
turistes durant els mesos d'estiu i una baixada en picat en l'entrada de
l'hivern/tardor. Així doncs, és necessària una diferenciació d'ordre 12
per eliminar aquest patró.

\includegraphics{ad_proj_st_files/figure-latex/unnamed-chunk-6-1.pdf}

S'observa que amb una diferenciació d'ordre 12 s'ha eliminat el patró
estacional. Ara bé, la mitjana de la sèrie encara no és constant.

\subsubsection{Mitjana constant}\label{mitjana-constant}

Per últim, es vol aconseguir que la sèrie tingui mitjana constant igual
(i si és possible igual a 0) per a poder considerar definitivament la
sèrie com un procés estacionari. Per aconseguir-ho, es realitzaran
diferenciacions regulars de la sèrie fins que s'obtingui el resultat
desitjat

\[ W_t = X_t - X_{t-1} = (1 - B)X_t \]

Es realitza la primera diferenciació. Els valors de mitjana i variància
aconseguits són els seguents:

\includegraphics{ad_proj_st_files/figure-latex/unnamed-chunk-7-1.pdf}

\begin{verbatim}
## [1] -0.0003272785
\end{verbatim}

\begin{verbatim}
##             V1
## V1 0.004266965
\end{verbatim}

Com es pot observar, la mitjana del procés diferenciat regularment un
cop es pot considerar constant i nula. Ara bé, es mira de diferenciar un
segon cop i s'observa que la variància augmenta i, per tant, es té
\emph{overdifferentiation}.

Es definitiva, la sèrie transformada pel logaritme, diferenciada un cop
i amb una diferenciació d'ordre 12 per eliminar el patró estacional
(\(\texttt{d1d12logserie}\)) és un procés estacionari de mitjana 0.

\includegraphics{ad_proj_st_files/figure-latex/unnamed-chunk-8-1.pdf}

\begin{verbatim}
## [1] 0.0001260592
\end{verbatim}

\begin{verbatim}
##           V1
## V1 0.0132293
\end{verbatim}

\subsection{ACF/PACF de les dades}\label{acfpacf-de-les-dades}

\subsection{Proposta de models}\label{proposta-de-models}

\section{Estimació dels models}\label{estimacio-dels-models}

\section{Validació dels Models}\label{validacio-dels-models}

\subsection{Estabilitat dels Models}\label{estabilitat-dels-models}

\section{Previsions a llarg termini}\label{previsions-a-llarg-termini}

\section{\texorpdfstring{Tractament de
\emph{outliers}}{Tractament de outliers}}\label{tractament-de-outliers}


\end{document}
